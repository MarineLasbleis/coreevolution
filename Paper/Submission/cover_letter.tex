% LaTeX Template
% Version 1.0 (1/6/13)
%
% This template has been downloaded from:
% http://www.LaTeXTemplates.com
%
% Original author:
% Matthew J. Miller
% http://www.matthewjmiller.net/howtos/customized-cover-letter-scripts/
%
% License:
% CC BY-NC-SA 3.0 (http://creativecommons.org/licenses/by-nc-sa/3.0/)
%
%%%%%%%%%%%%%%%%%%%%%%%%%%%%%%%%%%%%%%%%%

%----------------------------------------------------------------------------------------
%	PACKAGES AND OTHER DOCUMENT CONFIGURATIONS
%----------------------------------------------------------------------------------------

\documentclass[12pt,stdletter,dateno,sigleft]{newlfm}

\newlfmP{sigsize=50pt} % Slightly decrease the height of the signature field
%\newlfmP{addrfromphone} % Print a phone number under the sender's address
\newlfmP{addrfromemail} % Print an email address under the sender's address
\PhrPhone{Phone} % Customize the "Telephone" text
\PhrEmail{Email} % Customize the "E-mail" text

%\lthUiuc % Print the company/institution logo

%----------------------------------------------------------------------------------------
%	YOUR NAME AND CONTACT INFORMATION
%----------------------------------------------------------------------------------------

%\namefrom{Irene Bonati (corresponding author)} % Name

\addrfrom{
\today\\[12pt] % Date
Earth-Life Science Institute \\ % Address
Tokyo Institute of Technology \\
Meguro, Tokyo, Japan
}

\phonefrom{(000) 111-1111} % Phone number

\emailfrom{irene.bonati@elsi.jp} % Email address

%----------------------------------------------------------------------------------------
%	ADDRESSEE AND GREETING/CLOSING
%----------------------------------------------------------------------------------------

%\greetto{Dear Sir or Madam,} % Greeting text
%\closeline{Kindest Regards,} % Closing text

\nameto{Editorial office} % Addressee of the letter above the to address

\addrto{
JGR:Planets}

%----------------------------------------------------------------------------------------

\begin{document}
\begin{newlfm}

%----------------------------------------------------------------------------------------
%	LETTER CONTENT
%----------------------------------------------------------------------------------------

\textbf{Submission of a new article to the journal JGR:Planets}

\vspace{0.2cm}

Dear Sir or Madam,

We are submitting a manuscript, titled "Structure and thermal evolution of exoplanetary cores", by Irene Bonati, Marine Lasbleis, and Lena Noack for your consideration to be published as a research article in \textit{JGR: Planets}. 

This work presents new calculations investigating the internal structure in the aftermath of accretion, as well as the thermal and magnetic evolution of the cores of planets with different masses (0.8-2 Earth masses) and variable bulk and mantle iron inventories. Studying exoplanetary core and magnetic field evolution constitutes a first and mandatory step towards understanding the interactions and feedbacks between magnetic fields and planetary atmospheres, with the aim of constraining planetary interior properties from atmospheric properties. 

We find that while the planetary mass is a minor controlling parameter, the content and the distribution of iron substantially influence the size of the solid inner core and the lifetime of a magnetic field significantly. Even though iron-rich planets are able to produce stronger magnetic fields, their cores tend to become mostly or completely solid, which substantially shortens the field lifetime.

We are three early-career researchers, and we would be very happy if these results could be shared with the community as widely as possible. Due to our limited funding, we would like to ask for financial assistance to cover the open-access publication fee.

We are available to answer any questions about this submission. Please let us know if we may be of further assistance.

Thank you for your time and consideration. 

Kindest regards,

Irene Bonati (corresponding author)\\
Marine Lasbleis \\
Lena Noack
%----------------------------------------------------------------------------------------

\end{newlfm}
\end{document}